%% Encoding: ISO8859-1 %%

\chapter{Zusammenfassung und Ausblick}
\label{ch:Zusammenfassung}
% (Keine Untergliederung mehr!)
%% ==============================

In dieser Studienarbeit wurden die Grundlagen des Shamirs-Secret-Sharing-Verfahrens und das darauf basierende Naor-Pinkas-Revocation-Schema genauer erl�utert. Zudem wurde nach alternativen Interpolationsverfahren gesucht, die im Schema angewendet und ausgef�hrt werden konnten. Die zur Lagrange- sehr �hnliche Newton-Interpolation konnte in das Schema eingebaut werden, wobei der notwendige Schl�sselberechnungsaufwand eines Empf�ngers zu jedem Zeitpunkt h�her ausf�llt als bei der Interpolation mit Hilfe der Lagrange-Basisfunktionen.\\
\hspace*{0.45cm}Die approximative Interpolation mit Splines f�hrte zum Ergebnis, dass diese nicht ohne weiteres als Interpolationsm�glichkeit im Naor-Pinkas-Revocation-Schema angewendet werden kann. Bei der Umsetzung entstanden zwei Probleme. Das erste war die Sicherstellung von $x=0 \in [x_0,x_n]$, was relativ einfach behoben werden konnte, indem ein Dummy-Benutzer $x_d < 0$ hinzugef�gt wurde. Eine weitere L�sung ist die Verwendung eines anderen Schl�ssels $\mathcal S \in [x_0,x_n]$. Das zweite Problem konnte im Rahmen dieser Studienarbeit jedoch nicht ohne weiteres behoben werden. Es handelt sich hier bei um den Wertebereich des Schl�ssels $\mathcal S$. Als L�sungsvorschlag wurden eine Schwellenwertfunktion und Polynome von h�herem Grad in den Intervallabschnitten beschrieben, welche jedoch die Sicherheit des Revocations-Schemas reduzieren w�rden.\\
\hspace*{0.45cm}Im darauf folgenden Kapitel wurde das Asmuth-Bloom-Secret-Sharing-Schema genauer beschrieben. Das Ziel war, das Naor-Pinkas-Revocation-Schema einzubauen, um das Secret-Sharing-Schema als Broadcasting-Schema anwenden zu k�nnen. Da das Asmuth-Bloom-Secret-Sharing-Schema sehr �hnlich aufgebaut ist, wurde dies analog zu den �berlegungen der verschiedenen Interpolationsm�glichkeiten genauer betrachtet. In der Analyse fiel auf, dass der Einsatz des Naor-Pinkas-Revocation-Schema nicht ohne weiteres m�glich war. Grund hierf�r sind die vom Gruppencontroller berechneten Shares des Session-Schl�ssels. Diese k�nnen nicht von einem Empf�nger selber berechnet oder hergeleitet werden. Dieses Problem f�hrte aufgrund der Verlagung der Shares in den Exponenten nicht zum gew�nschten Ergebnis. In der Ver�ffentlichung von Kaya und Sel\c{c}uk \cite{ks07} k�nnen zu diesem Bereich weitere interessante �berlegungen nachgelesen werden.\\
\hspace*{0.45cm}Es existieren weitere Secret-Sharing-Verfahren, welche au�erhalb des Bereichs der Zahlentheorie und der Interpolationen liegen. Darunter f�llt auch die von Blakley \cite{b79} ver�ffentlichte Arbeit. In diesem Schema werden in einem $n$-dimensionalen Raum $n$ nichtparallele Hyperebenen definiert. Dadurch wird ein eindeutiger Schnittpunkt im Raum generiert, welcher jedoch nur mit Hilfe der $n$ Hyperebenen eindeutig ermittelt werden kann. Auch hier stellt sich die Frage, ob mit Hilfe des Naor-Pinkas-Revocation-Schemas die notwendigen Informationen in den Exponenten gezogen werden k�nnen, um auf diese Art und Weise ein Revocation-Schema zu erstellen.\\
\hspace*{0.45cm}Aufgrund der mathematischen Eigenschaften bieten Secret-Sharing-Verfahren eine gute M�glichkeit, den Schl�ssel vor unerlaubten Empf�ngern geheim zu halten. Mit ver�ffentlichten Arbeiten wie beispielsweise der von Naor und Pinkas werden derartige Schemata weiter verbessert und optimiert, um die Sicherheitsanforderungen der heutigen Welt gerecht werden zu k�nnen.
