
%% ==============================
\section{Vergleich der Interpolationsm�glichkeiten}
\label{ch:np-revocation-schema:sec:fazit-all}
%% ==============================

Wie bisher gezeigt wurde, k�nnen im Naor-Pinkas-Schema die Lagrange-Interpolation und die Newton-Interpolation gleicherma�en angewendet werden. In diesem Abschnitt sollen diese miteinander verglichen werden.\\
\hspace*{0.45cm}Diesbez�glich werden die Anzahl der verschiedenen Operationen (Addition A, Multiplikation M, Division D und Exponentiation E) der Lagrange- und Newton-Interpolation im Naor-Pinkas-Schema genauer untersucht.

\subsection{Additionen}
\label{sub:addition}
Die im Abschnitt \ref{ch:np-revocation-schema:sec:Aufwand} ermittelte Gesamtanzahl an Addition in der Lagrange-Interpolation im Naor-Pinkas-Schema betr�gt nach \eqref{eq:la_formel_gesamt}:
\[
  Anzahl_{L_A} = t\cdot(t+1)\,A
\]
Im Abschnitt \ref{ch:np-revocation-schema:sec:newton-overflow} wurde f�r die Newton-Interpolation folgende Anzahl an Additionen ermittelt:
\[
  Anzahl_{N_L} = \sum_{i=1}^t 2^{i-1}\cdot i\,A% = \frac{t(t+1)(2t+1)}{6}\,A
\]
Die folgende Tabelle zeigt sehr deutlich, wie sehr sich diese zwei Werte in Abh�ngigkeit von der Anzahl an unerlaubten Empf�ngern auswirkt:
\begin{center}
\begin{tabular}{c|c|c}
$|\mathcal R|$ & $Anzahl_{L_A}$ & $Anzahl_{N_A}$\\
\hline
10 & 110 & 9217\\
25 & 650 & 805306369\\
50 & 2550 & $5,517\cdot 10^{16}$\\
\end{tabular}
\end{center}
Die Anzahl an Additionen steigt in der Newton-Interpolation im Naor-Pinkas-Schema rasant an.

\subsection{Multiplikationen}
\label{sub:multiplikationen}
Aus \eqref{eq:la_formel_gesamt} ermitteln wird die Anzahl an Multiplikationen f�r die Lagrange-Interpolation im Naor-Pinkas-Schema:
\[
  Anzahl_{L_M} = \left(2\cdot(t+1)^2+t\right)\,M
\]
Im Vergleich dazu die aus der Newton-Interpolation:
\[
  Anzahl_{N_M} = \sum_{i=1}^t 2^{i-1}\cdot i\,M %= \frac{t(t+1)(2t+1)}{6}\,M
\]
Auch hier ist das Ziel nun, diese zwei Ergebnisse miteinander in Beziehung zu setzen:
\begin{center}
\begin{tabular}{c|c|c}
$|\mathcal R|$ & $Anzahl_{L_M}$ & $Anzahl_{N_M}$\\
\hline
10 & 252 & 9217\\
25 & 1377 & 805306369\\
50 & 5252 & $5,517\cdot 10^{16}$\\
\end{tabular}
\end{center}
�hnlich wie bei der Anzahl an Additionen steigt die Anzahl an Multiplikationen bei der Newton-Interpolation schneller an.

\subsection{Divisionen}
\label{sub:divisionen}
Entsprechend \eqref{eq:la_formel_gesamt} werden bei der Lagrange-Interpolation
\[
  Anzahl_{L_D} = (t+1)\,D
\]
Divisionen bei einem einzelnen Empf�nger ben�tigt, um das Polynom an der Stelle $g^{rP(0)}$ ermitteln zu k�nnen. Im Vergleich dazu werden
\begin{align*}
  Anzahl_{N_D} &= \sum_{i=1}^t \left( 2^i-1+2^{i-1} \right)\,D
  % \\
  % &=\left(\sum_{i=1}^t 2^i - \sum_{i=1}^t 1 + \sum_{i=1}^t i\right) \ D\\
  % &= (2^{t+1}-2) - t + \frac{t(t+1)}{2}\ D
\end{align*}
Anzahl an Divisionen bei der Newton-Interpolation gebraucht. Auch hier soll durch ein paar Werte gezeigt werden, dass diese Anzahl an beider Interpolationen sich stark voneinander unterscheiden:
\begin{center}
\begin{tabular}{c|c|c}
$|\mathcal R|$ & $Anzahl_{L_D}$ & $Anzahl_{N_D}$\\
\hline
10 & 11 & 3059\\
25 & 26 & 100663268\\
50 & 51 & $3,378\cdot10^{15}$\\
\end{tabular}
\end{center}

\subsection{Exponenten-Berechnungen}
\label{sub:exponentiation}
Bei der Lagrange-Interpolation werden
\[
  Anzahl_{L_E} = (t+2)\,E
\]
Exponenten-Berechnungen bei einem einzelnen Empf�nger ben�tigt, um das Polynom an der Stelle $g^{rP(0)}$ ermitteln zu k�nnen. Bei der Newton-Interpolation im Naor-Pinkas-Schema werden im Vergleich dazu
\begin{align*}
  Anzahl_{N_E} = (t+2 + \sum_{i=1}^t 2^{i-1})\,E
\end{align*}
Anzahl an Exponenten-Berechnungen bei der Newton-Interpolation gebraucht. Es ist leicht zu zeigen, dass $Anzahl_{L_E} < Anzahl_{N_E}$ gilt. Jedoch sollen auch hier ein paar Beispielwerten aufzeigen, wie schnell ein gro�er Unterschied beider Interpolationen entsteht:
\begin{center}
\begin{tabular}{c|c|c}
$|\mathcal R|$ & $Anzahl_{L_E}$ & $Anzahl_{N_E}$\\
\hline
10 & 12 & 1035\\
25 & 27 & 33554458\\
50 & 52 & $1,126\cdot10^{15}$\\
\end{tabular}
\end{center}

\subsection{Fazit}
\label{sub:fazit}
In den vorherigen Abschnitten wurde gezeigt, dass durch die notwendigen Umformungen in der Newton-Interpolation im Naor-Pinkas-Schema gegen�ber der Lagrange-Interpolation mehr Berechnungsschritte ben�tigt. Die Newton-Interpolation ist daher nicht besser geeignet. Die hier ermittelten Werte bei relativ kleinem $|\mathcal R|$ best�tigen dies. Der Unterschied spielt vor allem bei zustandslosen Empf�ngern mit beschr�nkter Speicherm�glichkeit eine gro�e Rolle und sollte nicht zu vernachl�ssigen sein.

% subsection fazit (end)

% \newpage
% Dabei wird besonders auf die arithmetischen Flie�kommaoperationen pro Sekunde ($FLOPS$ = Floating-point Operations per Second) eingegangen, wobei die �bliche Gewichtungen der Operationen in der folgenden Tabelle festgehalten werden:
% \begin{center}
% \begin{tabular}{l|l}
% \textbf{Gleitpunktoperation} & \textbf{Gewichtung} \\
% \hline
% Addition, Subtraktion, Multiplikation & 1 FLOPS\\
% Division & 4 FLOPS\\
% Exponentialfunktion & 8 FLOPS
% \end{tabular}
% \end{center}
% Werden die ermittelten Werte f�r die Lagrange-Interpolation aus \eqref{eq:la_formel_gesamt} genauer betrachtet, ergibt das
% \begin{align*}
% &\ \ \ \ \, (t^2+2t)\,M + t\cdot (t+1)\,D + t\cdot(t+1)\,A + (t+2)\,E\\
% &= (t^2+2t) + 4t\cdot (t+1) + t\cdot(t+1) + 8(t+2)\\
% &= 6t^2 + 15t + 16
% \end{align*}
% und aufgrund von \eqref{eq:newton-aufwand-gesamt} erh�lt die Newton-Interpolation die Gesamtgewichtung
% \begin{align*}
% &\ \ \ \ \left( \frac{t\cdot (t+1)}{2} \right)\,M + \left( \frac{t\cdot(t+1)}{2} \right)\,D + t\cdot(t+1)\,A+(t+1)\,E\\
% &= \frac{t\cdot (t+1)}{2} + 4\cdot \left( \frac{t\cdot(t+1)}{2} \right) + t\cdot(t+1) + 8(t+1)\\
% &= 3,5t^2 + 11,5t + 8\ .
% \end{align*}
% Das Ziel ist nun, diese zwei Ergebnisse zueinander in Beziehung zu setzen. Dabei wird mit einer Anzahl an unerlaubten Empf�ngern von $t>0$ folgende Absch�tzung gemacht:
% \[
%   6t^2 + 15t + 16 > 6t^2 + 15t + 8 > 6t^2 + 11,5t + 8 > 3,5 t^2 + 11,5t + 8
% \]
% Zusammengefasst bedeutet dieses Ergebnis, dass die Anzahl an notwendigen Operationen eines Empf�ngers bei der Newton-Interpolation im Vergleich zur Lagrange-Interpolation f�r jede beliebige Anzahl an unerlaubten Empf�ngern geringer ausf�llt. Dadurch kann das Naor-Pinkas-Verfahren mit Hilfe der Newton-Interpolation effizienter ausgef�hrt werden.

% Dass dieses Ergebnis eine nicht zu vernachl�ssigende Bedeutung hat, soll die folgende Tabelle darstellen.
% \begin{center}
% \begin{tabular}{c|c|c}
% \textbf{Unerlaubte Empf�nger $t$} & \textbf{Gesamtgewichtung Lagrange} & \textbf{Gesamtgewichtung Newton} \\
% \hline
% 100 & 61516 & 36158\\
% 1000 & 6015016 & 3511508\\
% 10000 & 600150016 & 350115008\\
% \end{tabular}
% \end{center}
% Ab einer gewissen Anzahl an unerlaubten Empf�ngern und mit Hilfe der oben definierten �blichen Gewichtung der Operationen w�chst der Unterschied der Anzahl an $FLOPS$ zwischen der Lagrange- und der Newton-Interpolation auf das $\frac{6}{3,5}$-fache an.
