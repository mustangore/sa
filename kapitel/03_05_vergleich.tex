
%% ==============================
\section{Vergleich der Interpolationsm�glichkeiten}
\label{ch:np-revocation-schema:sec:fazit-all}
%% ==============================

Wie bisher gezeigt wurde, k�nnen im Naor-Pinkas-Verfahren die Lagrange-Interpolation und die Newton-Inteprolation gleicherma�en angewendet werden. In diesem Abschnitt sollen diese miteinander verglichen werden. Dabei wird besonders auf die arithmetischen Flie�kommaoperationen pro Sekunde ($FLOPS$ = Floating-point Operations per Second) eingegangen, wobei die �bliche Gewichtungen der Operationen in der folgenden Tabelle festgehalten werden:
\begin{center}
\begin{tabular}{l|l}
\textbf{Gleitpunktoperation} & \textbf{Gewichtung} \\
\hline
Addition, Subtraktion, Multiplikation & 1 FLOPS\\
Division & 4 FLOPS\\
Exponentialfunktion & 8 FLOPS
\end{tabular}
\end{center}
Werden die ermittelten Werte f�r die Lagrange-Interpolation aus \eqref{eq:la_formel_gesamt} genauer betrachtet, ergibt das
\begin{align*}
&\ \ \ \ \, (t^2+2t)\,M + t\cdot (t+1)\,D + t\cdot(t+1)\,A + (t+2)\,E\\
&= (t^2+2t) + 4t\cdot (t+1) + t\cdot(t+1) + 8(t+2)\\
&= 6t^2 + 15t + 16
\end{align*}
und aufgrund von \eqref{eq:newton-aufwand-gesamt} erh�lt die Newton-Interpolation die Gesamtgewichtung
\begin{align*}
&\ \ \ \ \left( \frac{t\cdot (t+1)}{2} \right)\,M + \left( \frac{t\cdot(t+1)}{2} \right)\,D + t\cdot(t+1)\,A+(t+1)\,E\\
&= \frac{t\cdot (t+1)}{2} + 4\cdot \left( \frac{t\cdot(t+1)}{2} \right) + t\cdot(t+1) + 8(t+1)\\
&= 3,5t^2 + 11,5t + 8\ .
\end{align*}
Das Ziel ist nun, diese zwei Ergebnisse zueinander in Beziehung zu setzen. Dabei wird mit einer Anzahl an unerlaubten Empf�ngern von $t>0$ folgende Absch�tzung gemacht:
\[
  6t^2 + 15t + 16 > 6t^2 + 15t + 8 > 6t^2 + 11,5t + 8 > 3,5 t^2 + 11,5t + 8
\]
Zusammengefasst bedeutet dieses Ergebnis, dass die Anzahl an notwendigen Operationen eines Empf�ngers bei der Newton-Interpolation im Vergleich zur Lagrange-Interpolation f�r jede beliebige Anzahl an unerlaubten Empf�ngern geringer ausf�llt. Dadurch kann das Naor-Pinkas-Verfahren mit Hilfe der Newton-Interpolation effizienter ausgef�hrt werden.

Dass dieses Ergebnis eine nicht zu vernachl�ssigende Bedeutung hat, soll die folgende Tabelle darstellen.
\begin{center}
\begin{tabular}{c|c|c}
\textbf{Unerlaubte Empf�nger $t$} & \textbf{Gesamtgewichtung Lagrange} & \textbf{Gesamtgewichtung Newton} \\
\hline
100 & 61516 & 36158\\
1000 & 6015016 & 3511508\\
10000 & 600150016 & 350115008\\
\end{tabular}
\end{center}
Ab einer gewissen Anzahl an unerlaubten Empf�ngern und mit Hilfe der oben definierten �blichen Gewichtung der Operationen w�chst der Unterschied der Anzahl an $FLOPS$ zwischen der Lagrange- und der Newton-Interpolation auf das $\frac{6}{3,5}$-fache an.